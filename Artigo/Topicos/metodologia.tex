As metodologias aqui elencadas visam contribuir para especificar de forma prática as etapas do desenvolvimento do Sistema de Trilha Pedagógica Gamificada com Elementos de RPG. Espera-se que sejam aplicadas de forma prática nos mais diversos contextos pedagógicos e educacionais, armazenando e apresentando dados de forma acessível e informativa, facilitando o monitoramento e a tomada de decisões pelas equipes docentes envolvidas na aplicação e desenvolvimento desta ação, junto aos seus estudantes.
\\

\textbf{Desenvolvimento de Protótipos de Interface:} O Figma, uma plataforma de colaboração para design de interfaces e protótipos, será utilizada como ambiente colaborativo para desenvolver os protótipos da interface de usuário da aplicação. Isso permitirá que a equipe de design trabalhe de forma conjunta, compartilhando conceitos e recebendo feedback em tempo real, garantindo uma interface intuitiva e esteticamente atrativa para o Sistema de Trilha Pedagógica Gamificada com Elementos de RPG.
\\

\textbf{Planejamento do Modelo de Negócios:} O Business Model Canvas, uma ferramenta estratégica, será empregado no planejamento e esboço do modelo de negócios específico para o sistema de identificação e diagnóstico. Será utilizado para criar um modelo visual que descreva como o projeto se encaixa no contexto do monitoramento do avanço pedagógico dos estudantes. A equipe usará o Canvas para identificar parceiros estratégicos, recursos essenciais, atividades fundamentais e fontes de receita, mantendo a coesão com os objetivos do sistema.
\\

\textbf{Modelagem do Sistema de Software:} A UML (Unified Modeling Language) será aplicada como linguagem visual na modelagem do sistema de software. A plataforma utilizada para essa etapa sera o Lucidchart. Através de um estudo de caso, a UML será usada para representar elementos cruciais do sistema, como a interface de usuário, o fluxo de informações e as interações entre os módulos do sistema de identificação e diagnóstico do avanço pedagógico dos estudantes, contribuindo para uma compreensão completa e documentação eficaz da arquitetura do sistema.
\\

\textbf{Modelagem do Banco de Dados:} O DER (Diagrama de Entidade e Relacionamento) será adotado para modelar a estrutura do banco de dados do sistema, definindo entidades relevantes, como o público-alvo que utilizará o sistema (estudantes e professores), escala de proficiência, nível de proficiência do estudante, evolução de aprendizagem, habilidades a serem desenvolvidas, consolidadas ou aprofundadas, e nível de avanço do estudante e da sala de aula como um todo.
\\

\textbf{Estruturação do Banco de Dados:} Para a estruturação do banco de dados da aplicação, será utilizado o programa XAMPP, que possui um pacote com os principais servidores de código aberto, simulando o funcionamento pleno de consultas e conexões entre o sistema criado e o banco de dados, com suporte às linguagens PHP e Perl. Para a criação do banco de dados, optou-se pelo MySQL, por ser um sistema de banco de dados que utiliza a Linguagem de Consulta Estruturada (SQL - Structured Query Language), sendo de fácil utilização e integração com diferentes linguagens de programação.
\\

\textbf{Desenvolvimento do Sistema:} A equipe de desenvolvimento optou pelo editor de código-fonte Visual Studio Code, por possuir suporte de depuração, controle de versionamento Git incorporado, entre outros recursos que facilitam a codificação em Node.js, permitindo rodar códigos JavaScript fora dos navegadores. Este editor facilita a estruturação da aplicação, tanto no front-end quanto no back-end, em um único local. Nesta etapa, o foco do desenvolvimento será a versão desktop.
\\

\textbf{Criação dos Personagens:} Para a criação dos personagens, foi utilizada a plataforma Hero Forge, que permite projetar figuras de forma customizável, usando tecnologia de impressão 3D e na web. Isso permite visualizar as figuras em 3D e exportar imagens para a aplicação.
\\