No cenário educacional contemporâneo, a busca por metodologias inovadoras e eficazes que promovam o engajamento dos estudantes e facilitem a construção de conhecimento tem sido uma constante. Nesse contexto, a gamificação surge como uma abordagem que integra elementos de jogos em ambientes não lúdicos, como a sala de aula, com o objetivo de tornar o processo de aprendizagem mais dinâmico, motivador e significativo. Associada a essa tendência, a utilização de elementos de Role-Playing Game (RPG) se destacam como uma estratégia pedagógica promissora, capaz de estimular a participação ativa dos estudantes, o desenvolvimento de habilidades socioemocionais e a construção de conhecimento de forma transdisciplinar.
\\

Este artigo propõe uma reflexão sobre a aplicação de uma Trilha Pedagógica Gamificada com elementos de RPG como ferramenta para potencializar o ensino e a aprendizagem, incorporando os princípios do Desenho Universal da Aprendizagem (DUA). Partindo de uma análise das demandas educacionais atuais e das possibilidades oferecidas pelas Metodologias Ativas de Aprendizagem, o presente estudo explora como a integração de elementos típicos dos jogos de interpretação de papéis podem contribuir para a promoção de uma educação mais inclusiva, democrática e alinhada com as diretrizes curriculares nacionais, bem como os Objetivos de Desenvolvimento Sustentável (ODS).
\\

O ODS 4, que aborda a Educação de Qualidade, está diretamente relacionado à gamificação pedagógica, já que se concentra em garantir uma educação inclusiva, equitativa e de qualidade, promovendo oportunidades de aprendizagem ao longo da vida para todos. Por sua vez, o ODS 10, sobre a Redução das Desigualdades, aborda a necessidade de reduzir as disparidades sociais, econômicas e políticas dentro e entre países. A gamificação pedagógica pode ser uma ferramenta poderosa para promover a inclusão e a equidade, proporcionando oportunidades de aprendizagem acessíveis para todos os alunos, independentemente de sua origem socioeconômica ou situação. Por fim, o ODS 17, Parcerias e Meios de Implementação, destaca a importância da colaboração com outras instituições educacionais, organizações da sociedade civil e empresas. Ao trabalhar em conjunto, a gamificação pedagógica pode ampliar seu impacto e promover a implementação de todos os ODS.
\\

A proposta apresentada neste artigo surge da necessidade de desenvolver práticas pedagógicas que considerem a diversidade de habilidades, interesses e necessidades dos estudantes. Inspirada pelo paradigma do DUA (Desenho Universal da Aprendizagem), que busca garantir o acesso ao conhecimento a todos os estudantes, independentemente de suas características individuais, sendo elaborada uma abordagem gamificada que visa não apenas promover o aprendizado dos conteúdos curriculares, mas também desenvolver habilidades cognitivas, sociais e emocionais de forma equitativa.
\\

Além disso, destaca-se a importância da tecnologia educacional como um facilitador essencial nesse processo. A integração de recursos tecnológicos na Trilha Pedagógica Gamificada não apenas amplia as possibilidades de interação e personalização do ensino, mas também permite o acompanhamento individualizado do progresso dos estudantes, o compartilhamento de recursos e a criação de ambientes virtuais de aprendizagem colaborativa.
\\

Ao longo deste trabalho, serão discutidos os fundamentos teóricos da gamificação e do RPG como ferramentas educacionais, bem como serão apresentadas as possibilidades de aplicação do DUA na concepção e implementação da Trilha Pedagógica Gamificada. Além disso, serão explorados os benefícios da abordagem proposta para a promoção de uma educação inclusiva e personalizada, que atenda às necessidades de todos os estudantes. Em suma, este artigo busca contribuir para o debate sobre práticas inovadoras de ensino e aprendizagem, fornecendo subsídios teóricos e práticos para educadores interessados em explorar novas abordagens pedagógicas centradas no estudante.
\\

Com base nos documentos oficiais disponibilizados pelo Ministério da Educação – Instituto Nacional de Estudos e Pesquisas Educacionais Anísio Teixeira, utilizamos os relatórios e dados do Índice de Desenvolvimento da Educação Básica (IDEB) de 2021 , que consolida os resultados do Fluxo Escolar e as Médias de Desempenho nas Avaliações do Sistema de Avaliação da Educação Básica \href{https://download.inep.gov.br/educacao_basica/saeb/2021/resultados/relatorio_de_resultados_do_saeb_2021_volume_1.pdf}{SAEB} \citep{Brasil2022}. Esta análise permite projetar e compreender os avanços e retrocessos ao longo dos ciclos da Educação Básica, desde os estudantes do 5º Ano Ensino Fundamental Anos Iniciais, 9º Ano do Ensino Fundamental Anos Finais e 3ª Série do Ensino Médio.
\\

Outra fonte importante é o Sistema de Avaliação de Rendimento Escolar do Estado de São Paulo (SARESP), que fornece dados relevantes para monitorar políticas educacionais e direcionar aprimoramentos e projetos educacionais. A comparação entre os contextos amplo (estudantes das Redes Públicas da República Federativa do Brasil) e específico (estudantes da Rede Pública do Estado de São Paulo) é fundamental para nossa proposta. O Relatório de Resultados do SARESP (2021) aborda a Evolução da Aprendizagem, detalhando as médias de proficiência por ano/série e disciplina (\Cref{grph:EscProf}). Essa análise permite identificar quais resultados alcançaram o \href{https://download.inep.gov.br/publicacoes/institucionais/avaliacoes_e_exames_da_educacao_basica/escalas_de_proficiencia_do_saeb.pdf}{nível adequado de proficiência} e estimar as defasagens pedagógicas quando necessário (\Cref{grph:DistrPerc}).
\\
\begin{itemize}
\\
\begin{table}[!h]
\centering
\SetCaptionWidth{\ifbool{@LayoutA}{0.7}{0.72}\linewidth}
\caption{Escala de Proficiência (Língua Portuguesa)}%
\label{grph:EscProf}
\includegraphics[width = 1.3 \CaptionWidth]{tbEscalaProficiencia}
\SourceOrNote{Elaboração própria a partir do Boletim SARESP (2022).}
\end{table}
\\
\begin{table}[!h]
\centering
\SetCaptionWidth{\ifbool{@LayoutA}{0.7}{0.72}\linewidth}
\caption{Distribuição Percentual dos Alunos da Rede Estadual de São Paulo nos Níveis de Proficiência (Língua Portuguesa)}%
\label{grph:DistrPerc}
\includegraphics[width = 1.3 \CaptionWidth]{tbPercentualEstudantes.jpg}
\SourceOrNote{Elaboração própria a partir do Boletim SARESP (2022).}
\\
\end{table}
\\
\end{itemize}
\\

Para aprofundar a análise dos resultados do SARESP, a Secretaria da Educação do Estado de São Paulo desenvolveu o estudo "Evolução da Aprendizagem". Esse estudo tem como objetivo identificar as aprendizagens desejáveis e estender os Níveis de Proficiência aos anos/séries avaliados. Um destaque importante é a constatação de que estudantes da 3ª Série do Ensino Médio encerraram seus estudos com proficiência equiparável à dos estudantes do Nível Adequado do 8º Ano do Ensino Fundamental Anos Finais em Língua Portuguesa, apresentando assim, uma grande defasagem de aprendizagem.
\\

Portanto, as análises realizadas e a aplicação proposta terão como público-alvo os estudantes do Ensino Médio, com foco no desenvolvimento das Habilidades e Competências da Área de Linguagens na 1ª Série. Essa iniciativa visa atender a um público que demonstra familiaridade com jogos eletrônicos e uma disposição maior para participar de propostas gamificadas.
\\

A proposta da aplicação é envolver os estudantes em um Universo de Fantasia Histórica, que combina elementos fantásticos com períodos históricos reais, oferecendo uma perspectiva alternativa da história. Inicialmente focada no Brasil Colônia, entre os séculos XVI e XVIII, especialmente no Vale do Ribeira, e posteriormente expansível para todo o Estado de São Paulo, a escolha desse local e período se justifica pela colonização de Iguape em 1532, tornando-a a segunda cidade mais antiga do Brasil, e pela riqueza de biodiversidade, cultura e recursos naturais da região.
\\

O sistema se divide em duas interfaces, uma para os estudantes e outra para os professores, acessíveis tanto por aplicativo em smartphones/tablets quanto por dashboard em desktop/notebook, integrados ao Google Classroom. Isso permite a atualização do progresso dos estudantes e disponibilização de relatórios para os professores, auxiliando no alinhamento das atividades presenciais e online, estes resultados são convertidos em dados e disponibilizados aos professores para análise e intervenções futuras. Além disso, os estudantes têm a oportunidade de receber conquistas e prêmios por completarem desafios de forma satisfatória, e que pode incluir bônus temporários, itens virtuais ou até mesmo benefícios no mundo real, incentivando o engajamento, a corresponsabilidade e a educação integral. 
\\

A aplicação web será desenvolvida em Node.JS devido às suas vantagens, como leveza, rapidez e a capacidade de utilizar a mesma linguagem tanto no front-end quanto no back-end. Além disso, o Node.JS é reconhecido por sua ampla biblioteca, facilitando o desenvolvimento da aplicação de forma prática. É uma das plataformas mais utilizadas para interpretar código JavaScript no mercado. A sua arquitetura Modelo-Visão-Controlador (MVC) simplifica a estruturação dos códigos, organizando a aplicação em três camadas distintas: as "views", que proporcionam uma boa experiência ao usuário; os "models", responsáveis pela segura inclusão e manutenção dos dados; e os "controllers", que garantem maior confiabilidade no registro das informações. O Node.JS também oferece recursos como "partials", que funcionam como partições de informações exibidas nas páginas da aplicação. Isso garante uma fidelidade visual consistente, mesmo em telas com resoluções diferentes, mantendo a integridade da aplicação durante a expansão.
\\

O estudo está em andamento para determinar a melhor escolha para o desenvolvimento da aplicação móvel. Avaliando as necessidades do projeto, as habilidades da equipe de desenvolvimento e o perfil do público-alvo, a decisão será tomada de forma informada. Cada opção de plataforma de desenvolvimento será cuidadosamente considerada, levando em conta suas vantagens e desvantagens. Essa abordagem ajudará a garantir que a solução escolhida atenda efetivamente aos objetivos do projeto e às expectativas dos usuários.