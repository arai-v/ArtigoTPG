Com base nos apontamentos apresentados, a situação problema reside na busca por estratégias inovadoras e eficazes que promovam um aprendizado mais significativo e inclusivo no contexto educacional. Diante desse desafio, propõe-se a implementação de uma Trilha Pedagógica Gamificada com elementos de Role-Playing Game (RPG), que integra Metodologias Ativas de Aprendizagem. Essa abordagem visa engajar os estudantes, desenvolver habilidades e competências, e promover a construção interdisciplinar do conhecimento, alinhando-se às diretrizes curriculares nacionais e aos objetivos de uma educação integral.
\\

A proposta de solução abrange a criação de um Sistema de Trilha Pedagógica Gamificada com Elementos de RPG, aplicável em diversos contextos educacionais. Esse sistema facilitará o monitoramento e a tomada de decisões pelos docentes, promovendo uma educação mais personalizada e envolvente. Para tanto, serão desenvolvidos protótipos de interface no Figma, planejado o modelo de negócios com o Business Model Canvas, modelado o sistema de software com a UML e o DER, e estruturado o banco de dados com XAMPP e MySQL. O desenvolvimento do sistema ocorrerá no Visual Studio Code com Node.js, e os personagens serão criados na plataforma Hero Forge.
\\

O desenvolvimento meticuloso da aplicação representa um avanço significativo na integração da tecnologia e metodologias ativas de aprendizagem. Além de oferecer uma oportunidade para repensar e revitalizar o processo educacional, essa iniciativa capacita educadores e estudantes com uma ferramenta poderosa que pode catalisar a transformação do ensino e da aprendizagem. Ao fornecer uma interface intuitiva, um sistema de recompensas motivador e uma estrutura flexível, está se construindo as bases para um sistema educacional mais inclusivo, adaptável e eficaz.
\\

Portanto, o desenvolvimento minucioso e cuidadoso dessa aplicação é fundamental para o futuro da educação. Ao oferecer ferramentas e recursos inovadores que capacitam educadores e engajam os estudantes de maneira mais profunda e significativa, está-se dando passos importantes em direção a uma educação mais alinhada com as demandas do século XXI. A aplicação da Trilha Pedagógica Gamificada com Elementos de RPG representa um marco na evolução da prática educacional, promovendo um ambiente de aprendizado dinâmico, envolvente e adaptado às necessidades individuais dos estudantes.
\\










