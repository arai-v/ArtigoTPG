Este artigo busca traçar a evolução da gamificação na educação, desde suas raízes iniciais até seu estado atual, e examinar como essa abordagem pode ser efetivamente implementada para promover a aprendizagem e o engajamento dos estudantes na era digital. Ao refletir sobre o passado, podemos informar e inspirar as práticas futuras na integração bem-sucedida de jogos e educação.
\\

Os primeiros levantamentos sobre a utilização da metodologia de gamificação na educação remonta ao final dos anos 1990 e início dos anos 2000, liderado por Thomas W. Malone, do MIT, em colaboração com seus colegas. Em \citep{malone2021fazendo}, publicaram o influente artigo "Tornando a aprendizagem divertida: uma taxonomia de motivações intrínsecas para a aprendizagem", no qual discutiram a importância de motivar os estudantes de forma intrínseca para a aprendizagem, propondo uma taxonomia de motivações essenciais. Embora não seja explicitamente um estudo sobre gamificação na educação contemporânea, esse trabalho é frequentemente citado como uma das primeiras explorações acadêmicas da conexão entre motivação intrínseca e aprendizagem, aspectos cruciais para o desenvolvimento da gamificação educacional.
\\

De acordo com o artigo: " Dos elementos de design de jogos à ludicidade: definindo "gamificação", \citep{deterding2011game}, fornecem uma definição fundamental para o termo “gamificação”, explorando a transição dos elementos de design de jogos para o conceito mais amplo de “gamefulness”, abordando como os elementos de design presentes nos jogos podem ser aplicados em contextos não relacionados a jogos, como educação, saúde, marketing e ambiente de trabalho, para promover o engajamento e motivar o comportamento desejado.
\\

Já em "Um guia prático para a gamificação na educação", \citep{hsin2013practitioner}, apresenta um guia prático destinado a educadores interessados em implementar a gamificação em suas práticas de ensino. Este guia oferece insights valiosos sobre como usar a gamificação de forma eficaz na educação, fornecendo orientações e estratégias específicas para projetar e implementar experiências de aprendizado gamificadas. Ele visa ajudar os educadores a entender os princípios fundamentais da gamificação e como aplicá-los de maneira significativa em ambientes educacionais. 
\\

Em o estudo, “A gamificação funciona? - uma revisão da literatura de estudos empíricos sobre gamificação” \citep{hamari2014does}, revisa uma série de pesquisas empíricas sobre gamificação e seu impacto em diversas áreas, incluindo educação. Ao examinar o corpo de literatura existente, os autores investigam se a gamificação é eficaz em alcançar seus objetivos propostos, abordando questões como elementos de gamificação mais eficazes, contextos de sucesso e desafios enfrentados na implementação dessas estratégias.
\\

O artigo "Um teste empírico da teoria da aprendizagem gamificada: O efeito das tabelas de classificação no tempo gasto na tarefa e no desempenho acadêmico",  \citep{landers2014empirical} é um estudo empírico que investiga o impacto dos placares de líderes na motivação e desempenho dos estudantes. Especificamente, os autores examinam como a presença de placares de líderes, uma característica comum em ambientes gamificados, influencia o tempo dedicado à tarefa pelos estudantes e seu desempenho acadêmico.
\\

Em "Gamificação em teoria e ação: Uma pesquisa" \citep{seaborn2015gamification},  retoma a análise abrangente sobre a teoria e prática da gamificação. Reforça como a gamificação é aplicada em uma variedade de contextos, examinando tanto os fundamentos teóricos por trás dessa abordagem quanto exemplos práticos de sua implementação. O estudo aborda questões como motivação, engajamento do usuário e design de sistemas gamificados, além de discutir os efeitos da gamificação em diferentes áreas, incluindo educação, negócios e saúde.
\\

A correlação entre o uso da gamificação na educação e os índices de melhoria na qualidade educacional de um país é um tema complexo e multifacetado. Embora existam estudos e evidências que sugerem benefícios no engajamento dos estudantes e no desempenho acadêmico decorrentes da gamificação, é difícil atribuir diretamente essa prática a melhorias específicas nos índices educacionais de um país.  
\\

Alguns exemplos de países que têm explorado ativamente a gamificação na educação incluem os Estados Unidos, Finlândia, Singapura, Austrália, Reino Unido, Coreia do Sul e Canadá. Esses países frequentemente estão na vanguarda da inovação educacional e têm sistemas educacionais que incentivam a experimentação e o desenvolvimento de novas práticas pedagógicas com o uso de metodologias ativas de aprendizagem, e tendem a compartilhar algumas características comuns, como: 
\\

\begin{enumerate}
    \item \textbf{Investimento em Tecnologia Educacional:} Nações que investem significativamente em tecnologia educacional têm mais probabilidade de integrar a gamificação de forma eficaz. Isso inclui não apenas o acesso a dispositivos e infraestrutura tecnológica nas escolas, mas também o desenvolvimento de plataformas e aplicativos educacionais que incorporam elementos de gamificação;
\\
    \item \textbf{Políticas Educacionais Progressistas:} Países com políticas educacionais progressistas e flexíveis estão mais abertos a experimentar novas abordagens pedagógicas, incluindo a gamificação. Flexibilidade curricular e autonomia escolar podem permitir que as escolas implementem práticas inovadoras de ensino;
\\
    \item \textbf{Formação de Professores:} A formação adequada de professores é essencial para a implementação eficaz da gamificação na sala de aula. Países que investem em programas de desenvolvimento profissional para capacitar os educadores a integrar a gamificação em seu ensino estão mais bem posicionados para alcançar sucesso nessa área;
\\
    \item \textbf{Cultura de Inovação Educacional:} Nações com uma cultura de inovação educacional e uma mentalidade aberta à experimentação tendem a adotar mais prontamente abordagens como a gamificação. Isso pode ser incentivado por meio de políticas que apoiam a pesquisa e o desenvolvimento de novas práticas educacionais;  
\\
    \item \textbf{Engajamento dos Estudantes:} Países que priorizam o engajamento dos estudantes e reconhecem a importância do aspecto motivacional no processo de aprendizagem são mais propensos a adotar a gamificação como uma estratégia para tornar a educação mais envolvente e relevante para os estudantes.
\\
\end{enumerate}

Segundo o estudo intitulado: "Game Design: Plataforma Gamificada como Inovação Tecnológica na Educação" \cite{de2022game}, investigou o uso de plataformas gamificadas na educação online, especialmente no ensino básico, e mapeou os mecanismos de jogos mais utilizados. O estudo, caracterizado como exploratório, utilizou uma revisão da literatura para compreender o uso da gamificação na educação e uma análise de similares. Os resultados destacam a gamificação como uma tendência emergente na educação, identificando lacunas de pesquisa, como a necessidade de aplicação dos elementos de interação na educação, estudos de experiência do usuário e pesquisas a longo prazo. Este estudo pode beneficiar educadores em busca de plataformas gamificadas e profissionais envolvidos no desenvolvimento e aprimoramento dessas ferramentas.  
\\

Atualmente, algumas plataformas pedagógicas incorporam recursos de gamificação e elementos de Role Playing Game (RPG), mas nem todas conseguem integrar plenamente essas duas essências. Muitas delas negligenciam aspectos fundamentais do RPG, como a estruturação de personagens, a distribuição de Pontos de Atributos e Pontos de Habilidades. Esses elementos são cruciais para permitir que os jogadores personalizem e adaptem seus personagens de acordo com suas preferências e estratégias de jogo. Além disso, é observado que essas plataformas gamificadas geralmente não utilizam o recurso de "Quest" (missão ou tarefa), que é uma atividade ou objetivo atribuído aos personagens pelos narradores do jogo, pelo sistema do jogo ou por outros personagens não jogadores (NPCs). As quests desempenham um papel fundamental na narrativa e na jogabilidade de muitos RPGs, proporcionando aos jogadores objetivos específicos para superar desafios e conquitar recompensas.
\\

Com base nessas observações, três plataformas se destacam por fazer uso desses recursos:
\\

\\
\begin{enumerate}
    \item \textbf{Quest Atlantis:} Plataforma educacional baseada em jogos desenvolvida pela University of Indiana, nos Estados Unidos. Criada por Sasha Barab, Chris Dede e Kurt Squire, com contribuições de muitos outros pesquisadores e desenvolvedores, o projeto começou no final da década de 1990 e continuou a se desenvolver ao longo dos anos seguintes. O "Quest Atlantis" é um ambiente virtual de aprendizagem projetado para envolver os estudantes em experiências educacionais imersivas e interativas, permitindo que eles explorem diferentes temas e realizem missões enquanto aprendem conceitos acadêmicos;  
\\ 
\item \textbf{3DGameLab:} Plataforma de aprendizagem baseada em jogos que permite aos educadores criar cursos e atividades gamificadas para seus estudantes. Desenvolvido por Lisa Dawley e Chris Haskell na Boise State University, nos Estados Unidos, o projeto começou em 2011. Dawley e Haskell fundaram o "3D Game Lab" como um ambiente de aprendizagem inovador e envolvente para apoiar a educação baseada em jogos. A plataforma foi projetada para integrar elementos de jogos e gamificação no processo de ensino e aprendizagem, proporcionando uma experiência mais motivadora para os estudantes;
\\
\item \textbf{CassCraft:} Plataforma de gamificação para salas de aula desenvolvida por Shawn Young. O projeto começou em 2013, quando Young, um professor de física no Canadá, criou o "Classcraft" como uma maneira de tornar o aprendizado mais envolvente e motivador para seus estudantes. Desde então, a plataforma cresceu e se tornou uma ferramenta popular usada por educadores em todo o mundo para gamificar o ambiente de sala de aula e promover o engajamento dos estudantes. 
\\
\end{enumerate}

Com base na investigação conduzida sobre o tema proposto, "Trilha Pedagógica Gamificada com Elementos de RPG: Uma Abordagem para a Educação", e comparando-a com estudos anteriores realizados por outras linhas de pesquisa, observa-se que é viável desenvolver uma aplicação capaz de envolver tanto os educadores quanto os estudantes. Essa aplicação visa renovar práticas pedagógicas por meio do uso de metodologias ativas, que podem ser implementadas tanto de forma online quanto offline, de maneira síncrona ou assíncrona. 
\\

A proposta busca correlacionar o aprendizado dos estudantes, fornecendo dados e relatórios detalhados de seu desenvolvimento e progresso. Esses relatórios são apresentados por meio de gráficos e índices, permitindo que os professores compreendam o percurso de cada estudante e identifiquem suas necessidades específicas de intervenção, recuperação, consolidação ou aprofundamento de conhecimentos. Essa abordagem visa atender às demandas individuais de todos os estudantes envolvidos na ação educacional. 
\\




